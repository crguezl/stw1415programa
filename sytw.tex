\documentclass[11pt,a4paper]{article}
\usepackage[spanish]{babel}    
\usepackage[utf8x]{inputenc}
\usepackage{html} 

%%%%%%%%%%%%%%%%%%%%%%%%%%%%%%%%%%%%%%%%%%%%%%%%%%%%%%%%%%%%%%%%%%%%%%
%123456789012345678901234567890123456789012345678901234567890123456789
%%%%%%%%%%%%%%%%%%%%%%%%%%%%%%%%%%%%%%%%%%%%%%%%%%%%%%%%%%%%%%%%%%%%%%

%%%%%%%%%%%%%%%%%%%%%%%%%%%%%%%%%%%%%%%%%%%%%%%%%%%%%%%%%%%%%%%%%%%%%%
% Format 
%%%%%%%%%%%%%%%%%%%%%%%%%%%%%%%%%%%%%%%%%%%%%%%%%%%%%%%%%%%%%%%%%%%%%%

%\pagestyle{empty}
\pagestyle{plain} %each page number

%%%%%%%%%%%%%%%%%%%%%%%%%%%%%%%%%%%%%%%%%%%%%%%%%%%%%%%%%%%%%%%%%%%%%%
\title{
Sistemas y Tecnologías Web (2014-2015)
} 

\author{Casiano Rodriguez Leon}
%\date{\normalsize\today}

\begin{document}
\bodytext{BGCOLOR="#ffffff"}
\maketitle
\htmladdtonavigation{\htmladdnormallink{\htmladdimg[width=35 alt="pdf"]{gv.jpeg}}{sytw.pdf}}
\htmladdtonavigation{\htmladdnormallink{\htmladdimg[width=40 alt="SYTW UDV 14/15"]{moodlePP2logo.jpeg}}
{https://campusvirtual.ull.es/1415/course/view.php?id=5678}}
\htmladdtonavigation{\htmladdnormallink{\htmladdimg[width=35 alt="google"]{zip.jpg}}{sytw.zip}}

%1011: \htmladdtonavigation{\htmladdnormallink{\htmladdimg[width=40 alt="PP2moodle"]{moodlePP2logo.jpeg}}{http://campusvirtual.ull.es/1011/course/view.php?id=3276}}
%\htmladdtonavigation{\htmladdnormallink{\htmladdimg[width=50 alt="apuntes"]
%{camels.gif}}
%{http://nereida.deioc.ull.es/~pp2/perlexamples/}} 
%\htmladdtonavigation{\htmladdnormallink{\htmladdimg[width=35 alt="etsii"]{etsii.png}}{http://www.ull.es/view/centros/etsii/Inicio/es}}
%\htmladdtonavigation{\htmladdnormallink{\htmladdimg[width=35 alt="etsii"]{etsii.png}}{http://www.ull.es/view/centros/etsii/Ingeniero\_Superior\_en\_Informatica/es}}
%\htmladdtonavigation{\htmladdnormallink{\htmladdimg[width=35 alt="ull"]{ull.gif}}{http://www.ull.es/}}
%\htmladdtonavigation{\htmladdnormallink{\htmladdimg[width=35 alt="pcgull"]{logopcgull.gif}}{http://nereida.deioc.ull.es/}}

\section{Horarios}

    \noindent {\bf Tutorias}:

    \begin{itemize}
      \item Localización: Despacho \#96, 4ta Planta Edificio Física/Matem\'aticas. 
      \item Reserva: Tfno. 922 31 81 87 Fax. 922 31 81 70
      \item e-mail:
      \begin{htmlonly}
      \begin{rawhtml}
         <a href="mailto:casiano@ull.es">casiano at ull.es</a>
      \end{rawhtml}
      \end{htmlonly}
      \begin{latexonly}
         \verb#casiano@ull.es#
      \end{latexonly}
    \end{itemize}


\noindent {\bf Horario}:

\begin{enumerate}
\item 
Teoría y Problemas los tenemos en el Aula 1.2 los Jueves de 10:30 a 12:20. 
\item 
Grupo L2 los Viernes de 8:30 a 10:10 en la sala 2.3.
\item 
Grupo L1 los Viernes de 10:30 a 12:10 en la sala 2.3.
\item 
Tutorías de Lunes a Jueves de 09:00 a 10:30 en el despacho 96 de la 4ª planta del edificio de Física/Matemáticas
(crguezl@ull.edu.es, 922 31 81 87)
\end{enumerate}


Para la tutoría virtual contacta al profesor en google plus usando una "hangout". 
Mi identificador g+ de ull.edu.es es 101973567008975662199. También puede ir a
la página \htmladdnormallink{http://gplus.to/crguezl}{http://gplus.to/crguezl}.

Véase también el calendario oficial de la ULL en 

\htmladdnormallink{http://www.ull.es/Private/folder/institucional/ull/wull/estudios\_docencia/calendario/calendario\_grado201415.pdf}{http://www.ull.es/Private/folder/institucional/ull/wull/estudios\_docencia/calendario/calendario\_grado201415.pdf}.

\begin{rawhtml}
<b>Agenda de Sistemas y Tecnologías Web</b><p>
<iframe src="https://www.google.com/calendar/embed?mode=AGENDA&amp;height=400&amp;wkst=2&amp;hl=es&amp;bgcolor=%23ffffff&amp;src=ull.edu.es_fpnojq5g9sf5haao4r4ic56odo%40group.calendar.google.com&amp;color=%2328754E&amp;ctz=Atlantic%2FCanary" style=" border-width:0 " width="800" height="400" frameborder="0" scrolling="no"></iframe>"
\end{rawhtml}"

%-------------------------------------------------------------------------------
\section{Programa}

\begin{enumerate}
\item Diseño, desarrollo e implementación Web
\item Medios y librerías digitales
\item Arquitecturas orientadas a servicios
\item Fundamentos, sistemas, servicios y aplicaciones basados en contenidos multimedia
\item Software social y colaborativo
\item Comercio electrónico
\end{enumerate}


\section{Micro-Exámenes}
Aproximadamente cada cuatro semanas se realiza una prueba objetiva para la valoración de las actividades prácticas,
trabajos y proyectos que se califican de 0 a 10. 
Los alumnos deberán superar los micro-exámenes con una nota mínima de 5. 

Las fechas previstas son:


\begin{enumerate}
\item
Jueves 9 de Octubre
\item
Jueves 6 de Noviembre
\item
Jueves 18 de Diciembre
\end{enumerate}
%-------------------------------------------------------------------------------
\section{Presentaciones}

Todos los alumnos hacen al menos una presentación de un tema durante el curso Los 
alumnos deberán superar esta parte con una nota mínima de 5. 

El alumno realiza no mas de tres de estas exposiciones por curso. 
En la mayoría de los casos sobre el mismo tema en diferentes niveles de profundidad: básico, intermedio y avanzado.
No obstante, es posible cambiar de tema.


Sigue una lista de temas. Pueden proponer cualquier tema que encaje en el perfil de la asignatura. He evitado poner software propietario pero si alguien quiere proponerlo porque tiene acceso a el mismo, es bienvenido.

%\begin{enumerate}
%\item HTML
%  \begin{enumerate}
%  \item HTML 5
%  \item CSS
%  \item HAML http://haml-lang.com/ 
%  \item Ruby's ERB templating system http://rrn.dk/rubys-erb-templating-system
%  \end{enumerate}
%\item Frameworks. Si tienes un framework web en el que estés interesado y no figure entre los de la lista, no dudes en proponerlo.
%\begin{enumerate}
%\item PHP
%  \begin{enumerate}
%  \item CakePHP
%  \item CodeIgniter
%  \item Slim
%  \end{enumerate}
%\item Ruby
%  \begin{enumerate}
%  \item Rails 
%  \item Sinatra 
%  \item Padrino
%  \end{enumerate}
%\item Python
%  \begin{enumerate}
%  \item Django
%  \item Flask
%  \item Bottle
%  \item Zope
%  \end{enumerate}
%\item Java
%  \begin{enumerate}
%  \item Grails
%  \item Spring
%  \item WaveMaker
%  \item Spark
%  \end{enumerate}
%\item ECMAScript
%  \begin{enumerate}
%  \item Flex
%  \item MonoRail
%  \item DotNetNuke
%  \end{enumerate}
%\end{enumerate}
%\item Google 
%  \begin{enumerate}
%    \item Google App Engine
%    \item \htmladdnormallink{Google Apps Script}{https://developers.google.com/apps-script/}
%    \item Google Web Toolkit
%    \item \htmladdnormallink{Chrome Developer tools}{https://developers.google.com/chrome-developer-tools}
%  \end{enumerate}
%\item Javascript
%  \begin{enumerate}
%  \item Javascript
%  \item Coffescript 
%  \item Ajax
%  \item Node.js 
%  \item JQuery \htmladdnormallink{http://docs.jquery.com/}{http://docs.jquery.com/}
%  \end{enumerate}
%\item Testing Aplicaciones Web 
%  \begin{enumerate}
%  \item Selenium Grid 
%  \item Cucumber 
%  \end{enumerate}
%\item Deployment
%  \begin{enumerate}
%  \item Capistrano
%  \item Heroku http://www.heroku.com/ 
%  \end{enumerate}
%\item Autentificación. No dudes en proponer cualquier servicio de autentificación en el que estés interesado
%  \begin{enumerate}
%  \item SSL
%  \item CAS Central Authentication Service
%  \item OAuth  Open standard for Authorization.
%  \item Internet Authentication Service (IAS)
%  \end{enumerate}
%\item Redes Sociales. No dudes en proponer tu propio tema aqui.
%  \begin{enumerate}
%  \item Programación de Aplicaciones Facebook
%  \item Programación de Aplicaciones G+
%  \item Redes Sociales Distribuidas
%  \item Community Manager
%  \end{enumerate}
%\item Comercio Electrónico
%  \begin{enumerate}
%  \item Shopping cart software
%  \item Online Banking
%  \item Dinero Electrónico
%  \end{enumerate}
%\item Administración y Servicios Electrónicos, Gobierno Electrónico
%(cosas como SARA,
%red europea de Administración Electrónica TESTA,
%\verb|@|Firma,  \verb|TS@|, DNI electrónico, Plataforma de intermediación de datos,
%Red 060,
%Notificaciones Telemáticas Seguras,
%Portal de Administración Electrónica)
%\item Medios Digitales
%  \begin{enumerate}
%  \item Publicaciones electrónicas (e-books, epubs, podcasts, blogs, etc. )
%  \item Bibliotecas digitales
%  \item Modelos de Negocio en las Publicaciones Electrónicas
%  \item Tecnologías en las Publicaciones Electrónicas (Adobe, Apple, InformIT, etc.)
%  \end{enumerate}
%\item Programación Multimedia
%  \begin{enumerate}
%  \item PIL Python Imaging Library
%  \item Multimedia en iOS
%  \item Rubygame
%  \item SFML
%  \end{enumerate}
%\item Software Social y Colaborativo. Esto es, software diseñado 
%para ayudar a la gente a cooperar en la realización de  una tarea y a alcanzar unos objetivos que 
%son compartidos por los participantes.
%  \begin{enumerate}
%  \item Mensajería Instantánea (Como microsoft messenger)
%  \item Text Chat (IRC)
%  \item Software Colaborativo (groupware): Moodle, Sharepoint, Google, etc.
%  \item Foros, Wikis, Blogs, Redes sociales, Sistemas de Control de Versiones, etc.
%  \end{enumerate}
%\end{enumerate}


\begin{enumerate}
\item JavaScript, HTML y CSS
  \begin{enumerate}
  \item Objetos, 
  \item Funciones 
  \item Herencia
  \item Arrays
  \item Expresiones Regulares
  \item Métodos
  \item El DOM (Document Object Model)
  \item Eventos (Propagación, manejadores, contexto, etc.)
  \item Canvas
  \item SVG: Scalable Vector Graphics
  \item JavaScript en el lado del servidor: Node
  \item Audio y Vídeo
  \item Ajax
  \item JSON
  \item JQuery
  \item APIs HTML5 en JavaScript (p.ej. Geolocation, gestión de la historia, Web Sockets, etc.)
  \item BootStrap
  \item Testing/Pruebas en JS
  \item CSS
  \item Responsive Web Design
  \item SASS
  \item Web Design
  \item API de JavaScript de Google
  \item API de JavaScript de Google Maps
  \item AngularJS
  \item Raphael (JavaScript library to work with vector graphics on the web)
  \item Gráficos 3D
  \end{enumerate}
\end{enumerate}

%-------------------------------------------------------------------------------
\section{Proyectos}
Plantearemos proyectos en los que trabajarán en grupos de no mas de 3 personas.
%Se puede para ello "reciclar"  parte de un proyecto de Trabajo Fin de Grado.
%Se pueden usar las tecnologías que el grupo decida (PHP, Java, Python, Ruby, etc.)

El proyecto usará un sistema de control de versiones (git, hg, svn) y 
residirá en un repositorio público como GitHub, Google-Code o Bitbucket.

Cada semana habrá una exposición por el grupo de los avances logrados en el proyecto.

Para la primer semana tendremos que hacer una propuesta inicial de grupos y proyectos y ver que tecnologías necesitamos para poder instalarlas,
si es posible en el Centro de Cálculo. En caso contrario podremos trabajar con nuestros portátiles.

\section{Charlas}
Tendremos charlas de ex-alumnos y expertos en sistemas y Tecnologías Web a lo largo del curso que serán convenientemente anunciadas.
Si conoces de un ponente interesante no dudes en proponerlo.


También usaremos cursos y charlas en la web (Coursera, Udacity, etc.). En particular:
\begin{enumerate}
\item \htmladdnormallink{Coursera}{https://www.coursera.org}. \htmladdnormallink{Software Engineering for SaaS
}{https://www.coursera.org/course/saas} por Armando Fox y David Patterson
\item \htmladdnormallink{Udacity}{http://www.udacity.com}. \htmladdnormallink{Web Development (CS253).  How to Build a Blog}{http://www.udacity.com/overview/Course/cs253/CourseRev/apr2012} por Steve Huffman
\end{enumerate}

\section{Curso de Sinatra}
A lo largo del curso iremos desarrollando un curso de uso de Rack y Sinatra.

%-------------------------------------------------------------------------------
%-------------------------------------------------------------------------------
\section{Programa de Pr\'acticas}

Se llevarán a cabo micro-proyectos/prácticas individuales y en grupo cuya evaluación se hace mediante un 
taller/workshop. Los alumnos deberán superar los micro-proyectos con una nota mínima de 5. 

\section{Convocatorias de Exámenes}
Véase
\htmladdnormallink{Convocatorias de Exámenes del Grado para el curso 2014/2015 en http://www.ull.es/view/centros/etsii/}{http://www.ull.es/view/centros/etsii/Calendario\_de\_examenes\_3/es}


{\bf Enero}
\begin{verbatim}
DÍA: 12 de enero
HORA: 9:00
AULA: 2.1

DÍA: 20 de enero
HORA: 9:00
AULA: 2.1
\end{verbatim}

{\bf Junio}
\begin{verbatim}
DÍA: 27 de mayo
HORA: 16:00
AULA: 2.4
\end{verbatim}

{\bf Julio}
\begin{verbatim}
DÍA: 10 de julio
HORA: 9:30
AULA: 2.6
\end{verbatim}
%-------------------------------------------------------------------------------
%\bibliography{call}
\bibliographystyle{plain}
\begin{thebibliography}{99}

%\bibitem{bates} Bates Mark  \emph{Distributed Programming with Ruby}\\ 
%    Addison-Wesley 2010.
%
%\bibitem{rodpp2perl} Rodríguez C.  \emph{Programación Distribuida y Mejora del Rendimiento}\\ 
%    \begin{latexonly}
%    PDF en \verb|http://nereida.deioc.ull.es/~pp2/perlexamples/perlexamples.pdf|.\\
%    HTML en \verb|http://nereida.deioc.ull.es/~pp2/perlexamples/|.\\
%    \end{latexonly}
%    \begin{htmlonly}
%    PDF en \htmladdnormallink{http://nereida.deioc.ull.es/~pp2/perlexamples/perlexamples.pdf}{http://nereida.deioc.ull.es/~pp2/perlexamples/perlexamples.pdf}
%    HTML en \htmladdnormallink{http://nereida.deioc.ull.es/~pp2/perlexamples/}{http://nereida.deioc.ull.es/~pp2/perlexamples/}\\
%    \end{htmlonly}
%    2003.

\bibitem{definitive} Flanagan, David. \emph{JavaScript: The Definitive Guide}. O'Reilly. 2011.
\bibitem{crockford} Douglas Crockford. \emph{JavaScript: the Good Parts}. O'Reilly. 2008.
\bibitem{jumpstart} Ara Pehlivanian and Don Nguyen. \emph{Jump Start JavaScript}. SitePoint. 2013.
\bibitem{jumpstartcss} Louis Lazaris. \emph{Jump Start CSS}. SitePoint. 2013.
\bibitem{jumpstartnode} Don Nguyen. \emph{Jump Start Node.js}. SitePoint. 2012.
\bibitem{Fox} Fox, Armando; Patterson, David. \emph{“Engineering Long-Lasting Software. An Agile Approach Using SaaS \& Cloud Computing”}. Strawberry Canyon LLC. 2012.
\bibitem{Fowler} Fowler, Chad. \emph{“Rails Recipes”}. The pragmatic Bookshelf. 2011.
\bibitem{Marshall} Marshall, Kevin; Pytel, Chad; Yurek, Jon. \emph{“Pro Active Record: Databases with Ruby and Rails.”}. Apress. 2007
\bibitem{Griffiths} Griffiths, David, \emph{“Head First Rails: A Learner's Companion To Ruby On Rails”}. O’Reilly. 2009.
\bibitem{Al} Al Barazi, Rida; Carneiro Jr., Cloves; Carneiro, Cloves. \emph{“Beginning Rails 3”}. Apress. 2010.
\bibitem{Hartl} Hartl, Michael; Prochazka, Aurelius. \emph{“RailsSpace: Building a Social Networking Website with Ruby on Rails”}.  Addison-Wesley. 2007.
\bibitem{Bell} Bell, Gavin; \emph{“Building Social Web Applications”}. O’Reilly. 2009.
\bibitem{Follansbee} Follansbee, Joe. \emph{“Hands-On Guide to Streaming Media”}. Elsevier. 2006.
\bibitem{Christian} Christian; Laine, Jarkko, \emph{“Beginning Ruby on Rails E-Commerce: From Novice to Professional”}. Apress. 2006.
\bibitem{Dix} Dix, Paul, \emph{Service-Oriented Design With Ruby And Rails}. Addison-Wesley. 2010.
\bibitem{Richardson} Richardson, Leonard; Ruby, Sam, \emph{“RESTful Web Services”}. O’Reilly. 

\bibitem{rodlppruby} Rodríguez C.  \emph{Apuntes de la Asignatura Lenguajes y Paradigmas de Programación }\\ 
    \begin{latexonly}
    HTML en \verb|http://nereida.deioc.ull.es/~lpp/perlexamples/perlexamples.html|.\\
    \end{latexonly}
    \begin{htmlonly}
    HTML en \htmladdnormallink{http://nereida.deioc.ull.es/~lpp/perlexamples/perlexamples.html}{http://nereida.deioc.ull.es/~lpp/perlexamples/perlexamples.html}\\
    \end{htmlonly}
    2003.

\bibitem{vogel} Vogel, Lars.  \emph{Git Tutorial}
    \htmladdnormallink{http://www.vogella.de/articles/Git/article.html}{http://www.vogella.de/articles/Git/article.html}
    2012.

\bibitem{gitcommunity} The Git Community.  \emph{Git Community Book}
    \htmladdnormallink{http://book.git-scm.com/}{http://book.git-scm.com/}
    2012.

\bibitem{chacon} Scott Chacon.  \emph{Pro Git}
    \htmladdnormallink{http://progit.org/book/}{http://progit.org/book/}
    2010.

%\bibitem{rodperl} Rodríguez C.  \emph{Principios de Programación Imperativa, Funcional y Orientada a Objetos
%Una Introducción en Perl/Una Introducción a Perl}\\
%    \begin{latexonly}
%    Postscript en \verb|http://nereida.deioc.ull.es/~lhp/perlexamples.ps|.\\
%    HTML en \verb|http://nereida.deioc.ull.es/~lhp/perlexamples/|.\\
%    \end{latexonly}
%    \begin{htmlonly}
%    Postscript en \htmladdnormallink{http://nereida.deioc.ull.es/~lhp/perlexamples.ps}{http://nereida.deioc.ull.es/~lhp/perlexamples.ps}
%    HTML en \htmladdnormallink{http://nereida.deioc.ull.es/~lhp/perlexamples/}{http://nereida.deioc.ull.es/~lhp/perlexamples/}
%    \end{htmlonly}
%    2003.
%
%\bibitem{rodperl} Rodríguez C.  \emph{Análisis Léxico y Sintáctico}\\
%    \begin{latexonly}
%    Postscript en \verb|http://nereida.deioc.ull.es/~pl/perlexamples.ps|.\\
%    HTML en \verb|http://nereida.deioc.ull.es/~pl/perlexamples/|.\\
%    \end{latexonly}
%    \begin{htmlonly}
%    Postscript en \htmladdnormallink{http://nereida.deioc.ull.es/~pl/perlexamples.ps}{http://nereida.deioc.ull.es/~pl/perlexamples.ps}
%    HTML en \htmladdnormallink{http://nereida.deioc.ull.es/~pl/perlexamples/}{http://nereida.deioc.ull.es/~pl/perlexamples/}
%    \end{htmlonly}
%    2003.

%\bibitem{venables}
%Venables W.N., Ripley B.D. \emph{S Programming}. Springer. 2001.\\
%Programas usados en el libro:\\
%\htmladdnormallink{http://www.stats.ox.ac.uk/pub/MASS3/Sprog/Sprog.tar.gz}{http://www.stats.ox.ac.uk/pub/MASS3/Sprog/Sprog.tar.gz}.\\
%Copia en local de los programas usados en el libro:\\
%\htmladdnormallink{http://nereida.deioc.ull.es/\~{}pp2/R/Sprog.tar.gz}{http://nereida.deioc.ull.es/~pp2/R/Sprog.tar.gz}.\\
%Material del libro asociado \emph{Modern Applied Statistics with S-PLUS}:\\
%\htmladdnormallink{http://www.stats.ox.ac.uk/pub/MASS3/Sprog}{http://www.stats.ox.ac.uk/pub/MASS3/Sprog}.\\
%P\'agina con contenidos de ambos libros:\\
%\htmladdnormallink{http://www.stats.ox.ac.uk/pub/MASS3/}{http://www.stats.ox.ac.uk/pub/MASS3/}.\\

%\bibitem{Rteam}
%The R team. 2005.\\
%Introducción a R:\\
%\htmladdnormallink{http://nereida.deioc.ull.es/\~{}pp2/R/R-intro.html}{http://nereida.deioc.ull.es/~pp2/R/R-intro.html}.\\
%Extendiendo R:\\
%\htmladdnormallink{http://nereida.deioc.ull.es/\~{}pp2/R/R-exts.html}{http://nereida.deioc.ull.es/~pp2/R/R-exts.html}.\\
%Importando y Exportando datos desde y hacia R:\\
%\htmladdnormallink{http://nereida.deioc.ull.es/\~{}pp2/R/R-data.html.html}{http://nereida.deioc.ull.es/~pp2/R/R-data.html.html}.\\
%Instalación y Administración de R:\\
%\htmladdnormallink{http://nereida.deioc.ull.es/\~{}pp2/R/R-admin.html}{http://nereida.deioc.ull.es/~pp2/R/R-admin.html}.

%\bibitem{apuntes}
%Casiano R. Leon.
%Apuntes sobre Rendimiento.\\
%\htmladdnormallink{http://nereida.deioc.ull.es/\~{}pp2/performancebook/}{http://nereida.deioc.ull.es/~pp2/performancebook/}.\\

%\bibitem{gprof}
%Fenlason J., Stallman R. 
%GNU gprof.\\
%\htmladdnormallink{http://nereida.deioc.ull.es/\~{}pp2/gprof.html}{http://nereida.deioc.ull.es/~pp2/gprof.html}.\\

%\bibitem{foy}
%Brian D Foy.
%Benchmarking.
%The Perl Journal.
%Issue 11, Fall 1998 (volume 3, number 3).\\
%\htmladdnormallink{http://nereida.deioc.ull.es/\~{}pp2/Bench/tpjbench.html}{http://nereida.deioc.ull.es/~pp2/Bench/tpjbench.html}.\\

%\bibitem{goikhman}
%Mikhael Goikhman.
%Perl Threads.
%%\htmladdnormallink{http://nereida.deioc.ull.es/\~{}pp2/Perl\_Threads/}{http://nereida/~pp2/Perl_Threads/}.\\

%\bibitem{murray}
%Brad Murray and Ken Williams.
%Genetic Algorithms with Perl.
%The Perl Journal.
%Issue 15, Fall 1999 (volume 4, number 3).
%\htmladdnormallink{http://nereida.deioc.ull.es/\~{}pp2/Bench/tpjga.html}{http://nereida.deioc.ull.es/~pp2/Bench/tpjga.html}.\\

%\bibitem{lilja}
%Lilja, D.J. \emph{Measuring Computer Performance. A Practicioner's Guide}. Cambridge University press. 2000.

%\bibitem{Rod02}
%G.~Rodriguez Herrera.
%\newblock {CALL}: A complexity analysis tool.
%\newblock Master's thesis, Centro Superior de Inform\'atica. Univ. La Laguna,
%  2002.
%\newblock
%  \htmladdnormallink{http://nereida.deioc.ull.es/\~{}call}{http://nereida.deioc.ull.es/~call}.

%\bibitem{chandra}
%Chandra R., Dagum, L., Kohr D., Maydan D., McDonald J., Menon R. \emph{Parallel Programming in OpenMP}.
%Morgan Kaufmann Pub.  2001.

%\bibitem{paterson}
%Paterson D.A., Hennessy J.L. \emph{Estructura y Dise\~no de Computadoras}. Vol\'umenes 1 y 2. 
%Editorial Reverté. 2000.

%\bibitem{fortier}
%Fortier P.J., Michel H.E. \emph{Computer Systems. Performance Evaluation and Prediction}. 
%Digital Press.  Elsevier Science. 2003.

%\bibitem{CASIANO}
%    Rodríguez León, C. The Design, Analysis and Implementation of Parallel Algorithms for Shared Memory Machines\\
%    Postscript en \htmladdnormallink{http://nereida.deioc.ull.es/~pp2/openmpbook/openmp.ps}{http://nereida.deioc.ull.es/~pp2/openmpbook/openmp.ps}. 2002.
%    HTML en \htmladdnormallink{http://nereida.deioc.ull.es/~pp2/openmpbook/}{http://nereida.deioc.ull.es/~pp2/openmpbook/}. 2002.

%\bibitem{barry}
%Paul Barry. 
%\emph{Programming the Network with Perl}. 
%Wiley.

\end{thebibliography}

%-------------------------------------------------------------------------------

\end{document}
